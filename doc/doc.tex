%%%%%%%%%%%%%%%%%%%%%%%%%%%%%%%%%%%%%%%%%
% University/School Laboratory Report
% LaTeX Template
% Version 3.0 (4/2/13)
%
% This template has been downloaded from:
% http://www.LaTeXTemplates.com
%
% Original author:
% Linux and Unix Users Group at Virginia Tech Wiki 
% (https://vtluug.org/wiki/Example_LaTeX_chem_lab_report)
%
% License:
% CC BY-NC-SA 3.0 (http://creativecommons.org/licenses/by-nc-sa/3.0/)
%
%%%%%%%%%%%%%%%%%%%%%%%%%%%%%%%%%%%%%%%%%

%----------------------------------------------------------------------------------------
%	PACKAGES AND DOCUMENT CONFIGURATIONS
%----------------------------------------------------------------------------------------

\documentclass[12pt]{article}

\usepackage{mhchem} % Package for chemical equation typesetting
\usepackage{siunitx} % Provides the \SI{}{} command for typesetting SI units
\usepackage{amsmath}
\usepackage{listings}
\usepackage{commath}
\usepackage{epsfig}
\usepackage{verbatim}
\usepackage{graphicx} % Required for the inclusion of images
\usepackage[margin=1in]{geometry}
\usepackage{multirow}
\usepackage{color}
\usepackage{array}
\usepackage{epstopdf}

\setlength\parindent{0pt} % Removes all indentation from paragraphs

\renewcommand{\labelenumi}{\alph{enumi}.} % Make numbering in the enumerate environment by letter rather than number (e.g. section 6)

%\usepackage{times} % Uncomment to use the Times New Roman font

%----------------------------------------------------------------------------------------
%	DOCUMENT INFORMATION
%----------------------------------------------------------------------------------------

\title{MapReduce on Hadoop} % Title

\author{Jiahui \textsc{Guo}} % Author name

\date{\today} % Date for the report

\begin{document}

\maketitle % Insert the title, author and date


% If you wish to include an abstract, uncomment the lines below
% \begin{abstract}
% Abstract text
% \end{abstract}

%----------------------------------------------------------------------------------------
%	SECTION 1
%----------------------------------------------------------------------------------------

\section{Deploying Hadoop System on Hydra}
\subsection{Configure core site}
\begin{verbatim}
<configuration>
<property>
<name>hadoop.tmp.dir</name> 
<value>/local_scratch/jguo7/hadoop</value>
</property>
<property>
<name>fs.default.name</name> 
<value>hdfs://hydra22.eecs.utk.edu:55655</value>
</property>
</configuration>
\end{verbatim}

\subsection{Configure HDFS site}
\begin{verbatim}
<configuration>
<property> 
<name>dfs.replication</name> 
<value>3</value>
</property>
<property>
<name>dfs.name.dir</name>
<value>/local_scratch/jguo7/hadoop/</value>
</property>
<property>
<name>dfs.data.dir</name>
<value>/local_scratch/jguo7/hadoop/</value>
</property>
</configuration>
\end{verbatim}

\subsection{Configure the MapReduce site}
\begin{verbatim}
<configuration>
<property> 
<name>mapred.job.tracker</name>
<value>hydra22.eecs.utk.edu:55656</value>
</property>
</configuration>
\end{verbatim}

\subsection{Configure Master and Slaves}
\begin{itemize}
\item Master
    \begin{itemize}
    \item hydra22.eecs.utk.edu
    \end{itemize}
\item Slaves
    \begin{itemize}
        \item hydra23.eecs.utk.edu
        \item hydra24.eecs.utk.edu
        \item hydra26.eecs.utk.edu
    \end{itemize}
\end{itemize}
 
%----------------------------------------------------------------------------------------
%	SECTION 2
%----------------------------------------------------------------------------------------
\section{Building the Reverse-indexer}
\subsection{Identifying and Removing Stop Words}
\subsubsection{Pre-process Input File}
For the complete works of Shakespeare, the pre-processing step adds line numbers at the beginning 
of each line using the following command:\\
\verb+nl -nln pg100.txt > pg100_num.txt+\\
Then the pg100\_num.txt is used as the input file.

\subsubsection{Word-patterns to be Ignored}
The following word-patterns are added to "patterns.txt" file to be skipped.
\begin{verbatim}
\:
\, 
\.
\'
\;
\?
\!
\"
\-
\%
\^
\&
\*
\(
\)
\[
\]
\{
\}
\\
\/
\<
\>
\@
\#
\$
\^
\|
\'s
\end{verbatim}

\subsubsection{Stoping words}
After getting the result from the word counting, the words and its appearance time are listed 
in file ``sorted", The following shows a snippet of this file which helps to identify the threshold
for removing stoping words.
\begin{verbatim}
the 27578
and 26679
i   20682
to  19159
of  18120
a   14593
you 13615
my  12481
that    11506
in  10988
is  9568
not 8714
for 8217
with    7990
me  7769
it  7710
be  7075
your    6872
his 6857
this    6821
he  6552
but 6265
as  5951
have    5881
thou    5486
so  5254
him 5192
will    4975
what    4802
by  4380
thy 4032
all 3920
her 3843
are 3837
no  3790
do  3749
shall   3588
if  3493
we  3296
thee    3178
lord    3091
or  3076
on  3070
our 3061
king    3029
good    2812
now 2781
sir 2753
from    2642
o   2616
come    2507
at  2504
they    2471
well    2462
let 2356
she 2355
here    2323
which   2315
would   2293
more    2288
was 2229
then    2221
there   2200
am  2168
how 2161
love    2141
enter   2098
their   2075
when    2050
man 1993
them    1978
ill 1972
hath    1941
than    1880
may 1851
an  1833
one 1793
upon    1735
go  1733
like    1701
say 1680
know    1647
make    1631
did 1626
us  1619
such    1607
were    1577
should  1571
yet 1569
must    1491
why 1465
see 1439
had 1427
tis 1405
out 1376
where   1336
some    1332
give    1329
these   1322
time    1295
who 1287
too 1232
can 1208
------------------------
take    1197
most    1181
mine    1166
speak   1164
first   1160

\end{verbatim}
The dashed line indicates the threshold for the stoping word, the reason is most of the words above the threshold
appears at a relative high frequnecy and are modal verbs or pronoun, or even simple regular verb, which can not 
contain enough information and most important, such words could apprear in any documents with a high frequnecy. Thus
they are treated as ``noise" in this assignment. One thing needs to be noticed is that in this assignment the stemming
step is not performed, which means the program will treat ``book" and ``books" as different words.

\subsection{Building the Inverted Index}
\subsubsection{Removing stop words}
In the implementaion, the stop words are not physically removed, instead they are ``removed" in the processing of
building the key in the map stage using Java Regex. This is realized by adding the stop list into the ``patterns.txt"
file, padding with ``\textbackslash{}b" in order to match the exact word. For example, to remove the word ``book, they, we", the
following are added to ``patterns.txt'' file.
\begin{verbatim}
\bbook\b
\bwe\b
\bthey\b
\end{verbatim}

\subsubsection{Input file}
In order to demostrate the capability of multiple files, the input file ``pg100\_num.txt" is duplicated as 
\begin{verbatim}
pg1001
pg1002
pg1003
pg1004
\end{verbatim}
And these four files are used as input for building inverted index. 

\subsection{Query the Inverted Index}
The query built here is a local query, which could search the localtion of a specific word, if exists, the query will
return the filename and line offset; if not exists, the query will show ``the word does not exist!" 

\section{Demo}
Using the 4 input files, a snippet of the inversed index built is shown below:
\begin{verbatim}
zir->(pg1001, 62604)(pg1001, 62595)(pg1002, 62604)(pg1002, 62595)(pg1004, 62604)
(pg1004, 62595)(pg1003, 62604)(pg1003,62595)
zo->(pg1003, 62598)(pg1002, 62598)(pg1001, 62598)(pg1004, 62598)
zodiac->(pg1004, 108464)(pg1002, 108464)(pg1003, 108464)(pg1001, 108464)
zodiacs->(pg1001, 69627)(pg1002, 69627)(pg1004, 69627)(pg1003, 69627)
zone->(pg1003, 27750)(pg1002, 27750)(pg1001, 27750)(pg1004, 27750)
\end{verbatim}

The following shows how the query works and validate the accuracy of the query result.
\begin{verbatim}
$java Query -f part-00000 -w zone
localtion for word 'zone': 
(pg1003, 27750)(pg1002, 27750)(pg1001, 27750)(pg1004, 27750)

$java Query -f part-00000 -w zoa
The word zoa does not exist!
\end{verbatim}

%----------------------------------------------------------------------------------------


\end{document}
